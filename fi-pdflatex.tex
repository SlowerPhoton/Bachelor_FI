%%%%%%%%%%%%%%%%%%%%%%%%%%%%%%%%%%%%%%%%%%%%%%%%%%%%%%%%%%%%%%%%%%%%
%% I, the copyright holder of this work, release this work into the
%% public domain. This applies worldwide. In some countries this may
%% not be legally possible; if so: I grant anyone the right to use
%% this work for any purpose, without any conditions, unless such
%% conditions are required by law.
%%%%%%%%%%%%%%%%%%%%%%%%%%%%%%%%%%%%%%%%%%%%%%%%%%%%%%%%%%%%%%%%%%%%

\documentclass[
  digital, %% The `digital` option enables the default options for the
           %% digital version of a document. Replace with `printed`
           %% to enable the default options for the printed version
           %% of a document.
%%  color,   %% Uncomment these lines (by removing the %% at the
%%           %% beginning) to use color in the digital version of your
%%           %% document
  table,   %% The `table` option causes the coloring of tables.
           %% Replace with `notable` to restore plain LaTeX tables.
  twoside, %% The `twoside` option enables double-sided typesetting.
           %% Use at least 120 g/m² paper to prevent show-through.
           %% Replace with `oneside` to use one-sided typesetting;
           %% use only if you don’t have access to a double-sided
           %% printer, or if one-sided typesetting is a formal
           %% requirement at your faculty.
  nolof,     %% The `lof` option prints the List of Figures. Replace
           %% with `nolof` to hide the List of Figures.
  nolot,     %% The `lot` option prints the List of Tables. Replace
           %% with `nolot` to hide the List of Tables.
  %% More options are listed in the user guide at
  %% <http://mirrors.ctan.org/macros/latex/contrib/fithesis/guide/mu/fi.pdf>.
  draft=false,
  final,
]{fithesis3}
%% The following section sets up the locales used in the thesis.
\usepackage[resetfonts]{cmap} %% We need to load the T2A font encoding
\usepackage[T1,T2A]{fontenc}  %% to use the Cyrillic fonts with Russian texts.
\usepackage[
  main=english, %% By using `czech` or `slovak` as the main locale
                %% instead of `english`, you can typeset the thesis
                %% in either Czech or Slovak, respectively.
  english, german, russian, czech, slovak %% The additional keys allow
]{babel}        %% foreign texts to be typeset as follows:
%%
%%   \begin{otherlanguage}{german}  ... \end{otherlanguage}
%%   \begin{otherlanguage}{russian} ... \end{otherlanguage}
%%   \begin{otherlanguage}{czech}   ... \end{otherlanguage}
%%   \begin{otherlanguage}{slovak}  ... \end{otherlanguage}
%%
%% For non-Latin scripts, it may be necessary to load additional
%% fonts:
\usepackage{paratype}
% my packages
\usepackage{framed,color}
\usepackage{textcmds}
\def\textrussian#1{{\usefont{T2A}{PTSerif-TLF}{m}{rm}#1}}
%%
%% The following section sets up the metadata of the thesis.
\thesissetup{
    date        = \the\year/\the\month/\the\day,
    university  = mu,
    faculty     = fi,
    type        = bc,
    author      = Roman Solař,
    gender      = m,
    advisor     = {doc. RNDr. Pavel Matula, Ph.D.},
    title       = {Generating of Mathematical Problems for the Course PV189},
    TeXtitle    = {Generating of Mathematical Problems for the Course PV189},
    keywords    = {PV189, mathematics for computer graphics, SymPy, ROPOT question set, mathematical problem generation},
    TeXkeywords = {PV189, mathematics for computer graphics, SymPy, ROPOT question set, mathematical problem generation},
    abstract    = {%
      This thesis aims to generate question sets for the course PV189 while also providing a functional interface for future amendments and additions. The output must be compatible with IS ROPOTs, and it should be able to set a tolerance for non-integer answers. It should be possible to solve all the generated questions by hand, possibly with the aid of a calculator.  
The tasks generally revise basic concepts from the lectures and help build geometric intuition about them.  For many questions, symbolic computation has proven useful. In this case, the full functionality is implemented using Python 3; the symbolic computation is accomplished via the open-source Python library SymPy. 
    },
    thanks      = {%
      I take this opportunity to express gratitude to my supervisor doc. RNDr. Pavel Matula, Ph.D. and appreciate him for not losing his patience with me. Additionally, I would like to say a special thanks to Eva Drštková for going the extra mile to make this thesis possible. 
    },
    bib         = example.bib,
    %% Uncomment the following line (by removing the %% at the
    %% beginning) and replace `assignment.pdf` with the filename
    %% of your scanned thesis assignment.
%%    assignment         = assignment.pdf,
}
\usepackage{makeidx}      %% The `makeidx` package contains
\makeindex                %% helper commands for index typesetting.
%% These additional packages are used within the document:
\usepackage{paralist} %% Compact list environments
\usepackage{amsmath}  %% Mathematics
\usepackage{amsthm}
\usepackage{amsfonts}
\usepackage{url}      %% Hyperlinks
\usepackage{markdown} %% Lightweight markup
\usepackage{tabularx} %% Tables
\usepackage{tabu}
\usepackage{booktabs}
\usepackage[draft=false]{hyperref}
\usepackage{listings} %% Source code highlighting
\lstset{
  basicstyle      = \ttfamily,
  identifierstyle = \color{black},
  keywordstyle    = \color{blue},
  keywordstyle    = {[2]\color{cyan}},
  keywordstyle    = {[3]\color{olive}},
  stringstyle     = \color{teal},
  commentstyle    = \itshape\color{magenta},
  breaklines      = true,
}
\usepackage{floatrow} %% Putting captions above tables
\floatsetup[table]{capposition=top}
\thesissetup{assignment={}}
\begin{document}

\chapter{Introduction}

\shorthandoff{-}
\begin{markdown*}{%
  hybrid,
  definitionLists,
  footnotes,
  inlineFootnotes,
  hashEnumerators,
  fencedCode,
  citations,
  citationNbsps,
  pipeTables,
  tableCaptions,
}

The course PV189, Mathematics for Computer Graphics, takes place in fall, and its goal is to teach practical utilization of mathematics in the computer graphics area. It is obligatory for the master's program Visual Informatics at the Faculty of Informatics, Masaryk University. At the end of each lecture, students receive homework in the form of ROPOTs, practising the new topic. ROPOTs (acronym for **R**evision **O**pinion **PO**ll and **T**esting) is an online application (native to the Masaryk University information system IS) capable of assigning problems to students and evaluating their answers.

The project's goal is to automatically generate question sets compatible with ROPOTs, for various problems related to the lecture topic. It should be feasible to compute the question sets by hand on a sheet of paper, possibly with the help of a calculator.  

The project's outcome should also be a clean and reasonable way to manage the code for generating question sets. It has to be easily altered and expanded. 

The basis for the question sets is question sets created by Dmitry Sorokin, and Pavel Matula for the course PV189 \cite{ucebni_materialy}; concretely, I have been using the latest version from Fall 2020. Most of the original questions are only slightly modified, a few of them received a complete redesign. I drew the necessary theory to solve and generate question sets from lecture materials available at the same source. 

The questions are randomly generated, with nice numbers in the formulation preventing a tedious computation. Because the source code of the project responsible for the generation is public, I was forced to hide the parts of the code which are in control of the actual computation of the correct answer. To make it transparent, the sensitive parts of the code for each task have been cut out inside a method called \verb|solve_instance|. In the public version of the project, the body of this method for each implemented task has been replaced by the keyword `pass`. Therefore, this project version cannot be run and it would not be possible to check out its output. However, I have included a sample output for each of the implemented tasks in the appendix.

This thesis aims to introduce the tools I have used to create the project, document the project itself, and provide an easy-to-follow guide to use it. It briefly explains the contents of each lecture and lists all the implemented tasks. In order to understand the output of the project, it is necessary to have a basic knowledge of ROPOTs and their use in IS.   

\end{markdown*}
\shorthandon{-}


\chapter{Tools}

Throughout the implementation of the project, I had to familiarize myself with new tools in order to accomplish or facilitate different parts of the project. This includes well-known libraries \verb|numpy| and \verb|scipy| with plentiful documentation, and for this reason I will not expand on them further. Aside from these, I have frequently made use of the library for symbolic computation \verb|SymPy|, the basics of which are explained further in this section. 

As the purpose of this project is to create question sets compatible with the university information system IS, it is necessary to explain the basic syntax for creating question sets. 

\section{SymPy}

\shorthandoff{-}
\begin{markdown*}{%
  hybrid,
  definitionLists,
  footnotes,
  inlineFootnotes,
  hashEnumerators,
  fencedCode,
  citations,
  citationNbsps,
  pipeTables,
  tableCaptions,
}

To take advantage of the symbolic computation, the Python library SymPy \cite{sympy} proved to be a good fit for this project. It is versatile, and it supports most of the mathematical objects taught in the course PV189. Because of the symbolic computation, it is not very effective at calculating the results fast, but that is not a big issue in this project since we usually generate about 20 questions for each set. It is a handy tool for precise calculations, incorporating them into instances, and converting the expressions into \LaTeX{} to create questions.

Throughout my project, it is used in most tasks; namely, there are example methods using SymPy in the file \verb|sympy_utils.py| (and some also in \verb|utils.py|).  

\subsection{Introduction}

First, we need to import the library in order to use it.

```python
>>> import sympy as sp
```


There are many ways to create an expression in SymPy, depending on the type of expression we are trying to write. The simplest scenario might involve numerical simplifications.

```python
>>> sp.cos(sp.pi/3)
1/2
>>> sp.sqrt(8)
2*sqrt(2)
```

When we want to convert any SymPy expression into a \LaTeX{} string, we can call the \verb|sympy.latex| method.

```python
>>> sp.latex(sp.sqrt(8))
'2 \\sqrt{2}'
```

A more advanced scenario might use expressions with variables.

```python
>>> x = sp.Symbol('x')
>>> x**3 + 1**x + sp.exp(X)
x**3 + exp(x) + 1
```

The library is able to use these symbolic expressions for symbolic derivation, integration, simplification\ldots

```python
>>> sp.diff(x**3, x)
3*x**2
>>> sp.integrate(sp.cos(x) + 1, x)
x + sin(x)
```

Even these expressions can be turned into \LaTeX{} strings.

```python
>>> sp.latex(3*x**2)
'3 x^{2}'
```

\subsection{Usage in this project}

In this project, I am taking advantage of representing rational numbers in SymPy, because \verb|fractions.Fraction| can't be easily converted into \LaTeX{}, and it is necessary to keep precise rational representation for problem assignments. 

```python
>>> sp.Rational(3, 5)
3/5
>>> sp.latex(sp.Rational(3, 5))
'\\frac{3}{5}'
```

When it comes to printing, this library is also suitable for algebraic expressions. For example, when we want to use the implicit plane equation in the assignment ($ax + by +cz +d = 0$), it is complicated to do by hand. We must take into account that if $a = 0$, the whole term $ax$ must be left out, and so should be the $+$ sign after it. If the coefficient is 1 or -1, it should not be displayed at all. There are many such issues concerning this simple printing problem. With SymPy, it is easy because it has all this logic built-in already.

```python
>>> x, y, z = sp.Symbol('x'), sp.Symbol('y'), sp.Symbol('z')
>>> a, b, c, d = 0, -1, 1, 2
>>> sp.latex(a*x + b*y + c*z + d)
'- y + z + 2'
```

The library can work with complex numbers, and it is able to simplify them. 

```python
>>> (1 + sp.I)**2 
(1 + I)**2
>>> ((1 + sp.I)**2).simplify()
2*I
```

Every expression in SymPy has the \verb|simplify| method, which is highly customizable and can be forced into simplifying only certain aspects of the expression -- such as trigonometric, power, or  factor simplifications. It can also be suppressed.

```python
>>> sp.sqrt(8, evaluate=False)
sqrt(8)
```

We can even specify parts of the expression that must not be simplified. This is especially useful in assignments, where the students need to simplify the given expressions themselves.

```python
>>> (x + 1) * 2
2*x + 2
>>> sp.UnevaluatedExpr(x + 1) * 2
2*(x + 1)
```

SymPy is able to work with quaternions, even though it is very limited in certain aspects. It supports quaternion multiplication, addition, and integer exponentiation.  

```python
>>> sp.Quaternion(0, 1, 2, 3)
0 + 1*i + 2*j + 3*k
>>> sp.Quaternion(0, 1, 2, 3) * sp.Quaternion(0, 1, 0, 0)
(-1) + 0*i + 3*j + (-2)*k
>>> sp.Quaternion(0, 1, 0, 0) ** 2
(-1) + 0*i + 0*j + 0*k
```

The library also supports quaternion logarithm, but it is not very intuitive or convenient. It can be computed using the protected method \verb|\_ln| of the \verb|Quaternion| instance. See \ref{subsection:pitfalls} for more information.

```python
>>> sp.Quaternion(0, 1, 0, 0)._ln()
0 + pi/2*i + 0*j + 0*k
```

\subsection{Pitfalls}
\label{subsection:pitfalls}

Aside from having many benefits for this project, the SymPy library is far from perfect. It works fine for basic expressions, but we can find many imperfections if we dive deeper into the library. For example, the native logarithm method does not support quaternions. It treats it as a symbolic logarithmic expression.

```python
>>> sp.ln(sp.Quaternion(0, 1, 0, 0))
log(0 + 1*i + 0*j + 0*k)
>>> sp.ln(sp.Quaternion(0, 1, 0, 0)).simplify()
log(0 + 1*i + 0*j + 0*k)
>>> sp.ln(sp.Quaternion(0, 1, 0, 0)).evalf()
log(0 + 1*i + 0*j + 0*k)
```

SymPy only supports integer exponentiation for quaternions, even though it could be easily implemented using the built-in \verb|exp| and \verb|\_ln| methods. For some assignments, I needed to implement this functionality myself (this can be found in the \verb|sympy_utils.py| file).

The SymPy documentation is not complete; the methods are often not described in enough detail (e.g., the method `Quaternion.to_axis_angle`); for some methods, the documentation is missing altogether (e.g., \verb|Quaterion.\_ln|). I often had to read the actual implementation of the library to understand its behavior at edge cases. Luckily, the source code is meticulously documented, and it is available online \cite{sympy_source}. 

The library does not always produce the best \LaTeX{} strings for all expressions. It is customizable, but it takes a lot of effort, and ideally, it should be the default.

```python
>>> sp.latex(sp.Quaternion(0, 1, 0, 0))
'0 + 1 i + 0 j + 0 k'
>>> sp.latex(sp.Quaternion(0, -3, 0, 0))
'0 + \\left(-3\\right) i + 0 j + 0 k'
```

Also, this library does not have a convenient way to work with vectors. They are treated as matrices, so it is necessary always to carry the extra dimension. An alternative is to use the `CoordSys3D` class, but it has its own inconveniences because it represents the coordinates of the vector as attributes `i, j, k` of the class. It also supports vectors in three dimensions only.

\end{markdown*}
\shorthandon{-}

\section{ROPOTs}

\shorthandoff{-}
\begin{markdown*}{%
  hybrid,
  definitionLists,
  footnotes,
  inlineFootnotes,
  hashEnumerators,
  fencedCode,
  citations,
  citationNbsps,
  pipeTables,
  tableCaptions,
}

Because the outcome of this project is question files, which can be pasted directly into IS, I find it necessary to explain the basic ROPOT syntax I need to use in my project. 

Let's take a look at a simple ROPOT file.

```
What is 3 * 2.5? :n
What is bigger: apples or blueberries? :t

:n=7.5 ok
:t=apples ok
```

In this example, the first occurrence of `:n` creates an input field for the student to put the answer into. The letter `n` specifies that the input will be a number. Similarly, the first occurrence of `:t` creates an input text field.

Afterward, we need to declare the correct answer to these fields somewhere. It is done by repeating the name of the field and using the equals sign `=` to assign the correct answer to it. There must be no whitespace around the equals sign. Each declaration must be on its own line, and the line must end with `ok` to confirm the correctness of the answer. This might seem excessive, but it is useful for creating a list of possible answers for the student.

If we need to include multiple input fields of the same type in the question, we can differentiate them by numbering them.

```
What is 3 * 2.5? :n1
What is 3 * 3?   :n2 

:n1=7.5 ok
:n2=9   ok
```

Naturally, there can be multiple questions inside a question set. They are separated by a single line containing precisely two dashes. The input field names exist within a single question only, and thus we can reuse them in another question.

```
What is 3 * 2.5? :n1
What is 3 * 3?   :n2 

:n1=7.5 ok
:n2=9   ok
----
What is 2 * 2.5? :n1
What is 2 * 3?   :n2 

:n1=5 ok
:n2=6 ok
```

Using the input number field, we can even specify a range in which the answer is considered correct. The range is specified using two dots~`..`, and the answer must be enclosed in double-quotes. Any answer within the specified range (inclusive) will be accepted as correct.

```
What is 3 * 2.5? :n1
What is 3 * 3? :n2

:n1="7.4..7.6" ok
:n2=9 ok
```

The questions can use \LaTeX{} to format algebraic expressions. In that case, the math must be enclosed inside `<m>` and `</m>` tags. 

```
What is <m>3 \times2.5</m>? :n

:n="7.4..7.6" ok
```

Finally, we can specify the number of points the student gets for the correct answer. To do this, we insert the number of points right after the definition of the correct answer. In this case, we want to reward the correct answer with 1 point.

```
What is <m>3 \times2.5</m>? :n

:n="7.4..7.6" 1 ok
```

There are many more ways to customize a question, and they are documented in IS \cite{ropots_def}. For this project, more advanced syntax is not necessary.

\end{markdown*}
\shorthandon{-}

\chapter{Using the project}

\shorthandoff{-}
\begin{markdown*}{%
  hybrid,
  definitionLists,
  footnotes,
  inlineFootnotes,
  hashEnumerators,
  fencedCode,
  citations,
  citationNbsps,
  pipeTables,
  tableCaptions,
}

There are two basic scenarios in which you would need to work with the project. The first one is when you use it to generate questions, e.g., when you want to update a set of questions because the students have memorized the answers already. In order to do it, you need to know the structure of the project. In the second scenario, you create or alter a task. In that case, you need to understand the structure of the project as well as implementation details. You should know where to put the new task, what you need to fill it with, and how to do it in accordance with the project best practices. 

\section{Generating question sets}

*Note that this section does not aim to dive deep into details; instead, it provides a quick overview of the standard workflow. It is handy if you already have a basic understanding of the provided methods.*

You want to generate a new question set for the desired task. You need to know what task this is and the lecture it is in. 

1. In the project tree, navigate to the folder corresponding to the lecture covering the task you want to work with.
    
2. Find the python file corresponding to the task inside the folder. The filename should match the task name.
    
3. Execute the file. This will load all necessary dependencies and the class corresponding to the task. The class name is equal to the task name. 
    
4. Once the class is loaded, you can generate the questions calling its methods. It is recommended to call
``` Python
>>> TaskClassName.prepare_n_questions(n,  limits, tolerance, filename, points)
```
where you replace ```TaskClassName``` with the actual name of the class. 
    * argument **```n```** specifies the number of questions to be created
    * argument **```limits```** specifies the limits for the generated questions
    * argument **```tolerance```** specifies the tolerance for the numerical answers (typically 0.0001)
    * argument **```filename```** specifies the path to a file to store the generated questions
    * argument **```points```** specifies the number of points for each answer field (as a list of points respecting the order of the answer fields)
    
#. You can find the generated questions inside the file **```filename```**. Copy and paste them directly inside a question set in IS.

\subsection{Quick example}

Here is a quick Python console example for generating a question set for a concrete task (`ScalarProductFromAngle`).  

```Python
>>> ScalarProductFromAngle.prepare_n_questions(20, [-5, 5], 0.0001,                         "scalar_product_from_angle_questions.txt")
```

This command generates 20 questions inside the file \newline `scalar_product_from_angle_questions.txt`. The tolerance for the answers is 0.0001 and the limits are `[-5, 5]`, the points are not specified so the task default will be used. 

\section{Testing task implementation}

You want to make sure the implemented task produces valid instances. 

1. In the project tree, navigate to the folder corresponding to the lecture covering the task you want to work with.
    
2. Find the python file corresponding to the task inside the folder. The filename should match the task name.
    
3. Execute the file. This will load all necessary dependencies and the class corresponding to the task. The class name is equal to the task name. 
    
4. Once the class is loaded, you can generate a list of instances calling its methods. Save the result inside a variable to use it later.
``` Python
>>> instances = TaskClassName.get_n_unique_instances(n, limits, max_rounds=100)
```
where you replace ```TaskClassName``` with the actual name of the class. 
    * argument **```n```** specifies the number of different instances to be created
    * argument **```limits```** specifies the limits for the generated instances
    * argument **```max_rounds```** specifies the maximal number of unsuccessful attempts to retrieve a different instance in a row (the instances are typically randomly generated)
    * the method returns a list of `n` unique instances

5. Now, it is possible to test the validity of the generated instances. 
``` Python
>>> TaskClassName.test_instances(instances)
```
where you replace ```TaskClassName``` with the actual name of the class. 
    * argument **```limits```** takes a list of instances to be tested
    * the method prints out the number of invalid instances, and the concrete instances that failed the tests
    * the method returns `True` if all the instances are valid, `False` otherwise


\subsection{Quick example}

Here is a quick Python console example for testing a concrete task (`ScalarProductFromAngle`). 

```Python
>>> instances = ScalarProductFromAngle.get_n_unique_instances(100, [-5, 5])
>>> ScalarProductFromAngle.test_instances(instances)
0 errors in 100 instances
True
```


\chapter{Development guide}

You should read this chapter if you want to implement your own task or understand the very details of the project implementation. This is the main documentation of the project, but you can find more documentation inside the project files and the project \verb|README|. This might help, if something seems unclear. As a last resort, you can always check the project implementation.

\section{Creating a new task}

Before creating a new task, make sure you understand the difference between a question and a task. You should know what an instance is and how methods work with it. If this is not the case, I recommend starting out with \hyperref[sec:example]{the example}. 

\subsection{Updating the project tree}

#. Start by creating a **new file** for the task in the folder corresponding to the lecture
the task belongs to. (Folder name '*LX*' stands for '*Lecture X*'.) 
\definecolor{shadecolor}{rgb}{1,0.8,0.3}
\begin{shaded}{**Best practice:**}
    The strict rule is one task per one file.
\end{shaded}

#. Create a **new class** in the file to represent the task. The class must be a child of
the common parent class `Task` (located in the root of the project in `task.py`).

#. Choose a descriptive **name for the task**.
\definecolor{shadecolor}{rgb}{1,0.8,0.3}
\begin{shaded}{**Best practice:**}
    The name of the class should be in the `CapitalizedWords` style, the **name of the file** should be the same but in the `lower_case_with_underscores` style.
\end{shaded}

\subsection{Implementing abstract methods}

The listed abstract methods are obligatory to implement. If something is missing here,
you can check out the corresponding documentation in the class `Task`. 
It is recommended to implement the methods in the described order.

\subsubsection{Abstract method genfun} 

This is the core function of the task, it is responsible for **creating a valid
instance** of the problem. The instance of the problem is a term coined to describe
the concrete values of the variables of the problem and its solution.

The only parameter of the method, `limits`, can be used to **limit the randomly
generated values** of the problem variables. There are no restrictions on the
parameter --- it can be a list of numbers, a dictionary... However, in almost 
every case, it is a list of two numbers, where the first number represents the 
minimum value, and the second number the maximum value for a random generator.

\definecolor{shadecolor}{rgb}{1,0.8,0.3}
\begin{shaded}{**Best practice:**}
    Sometimes, it is necessary to leave out some numbers from the range (such as 0 or negative numbers).
In this case, the method should be robust enough to ignore undesired values within the range.
\end{shaded}

The function **returns a valid instance** of the problem, typically randomly generated.
However, the instance **must be hashable** (it has a `__hash__()` method) and so all lists inside an instance should be replaced by tuples etc.


\subsubsection{Abstract method testfun} 

This method takes an **instance** as a parameter and tests if it is **valid**. Since this method is not generally used when generating questions, the developer has a much more free hand implementing this method.

In general, the tests should sanity check the values in the instance (e.g., when asked to compute the division of two numbers `a` and `b`, we check that `b` is not equal to 0), and check that the result is compatible with the other values. Naturally, checking the latter should be done using different techniques (typically different libraries) than inside `genfun` or `solve_instance`.

The method should ideally **never fail** or raise an exception, instead it **returns**
`True` if the instance is valid and `False` otherwise.

Sometimes, it is impossible to meet the aforementioned criteria~---~usually because there is no other widespread trustworthy library compatible with the format of the instance. In that case, the test can either be partial or completely skipped. In case you want to skip implementing the method `testfun`, make it always return `True` and provide a comment explaining why you have decided to do so.

\subsubsection[Abstract method create_question]{Abstract method `create_question`} 

This method takes a valid instance (returned by `genfun`) and turns it into a question,
which can be passed into a ROPOT in IS. 

In order to be able to do this, it must know the **tolerance** for the answer (typically $0.0001$),
and the **points** given for each answer (there might be multiple input fields).

It **returns a string** which contains the question for the ROPOT.

\subsection{Finishing the implementation}

*This section only applies to you if you intend to make your changes to the project public.*

Because in the final version, there should be no easy guideline
for students to solve the given questions, we need to "hide" the
result-computing logic of the `genfun` method inside a new static method 
`solve_fun`. In the published version, the body of this new method will be 
replaced with the keyword `pass`.

\definecolor{shadecolor}{rgb}{1,0.8,0.3}
\begin{shaded}{**Best practice:**}
Before you finish the implementation of the new task, you should document it. Each task class documentation should explain
\begin{itemize}
    \item the task itself (what problem the student is supposed to solve),
    \item what an instance of the task is composed of and 
    \item the role of the argument `limits`.
\end{itemize}
\end{shaded}

\section{Example}
\label{sec:example}

*In this section, I will use a very primitive example to demonstrate how the 
development of a new task should ideally work.*

The example problem is to find the following number in the arithmetic sequence, 
given by its first two terms.

### Creating a new task

The students should have mastered this topic already, and we do not need to cover 
it in any lecture. Therefore, we will create a new folder `L0` and place a new
file `arithmetic_sequence.py` in it. Inside the file, I will create the following template.

```python
import random as rnd

from task import Task

class ArithmeticSequence(Task):

    @classmethod
    def create_question(cls, instance, tol, points):
        pass
    
    @classmethod
    def genfun(cls, limits):
        pass
    
    @classmethod
    def testfun(cls, instance):
        pass
```

### Implementing abstract methods

First, I will implement the `genfun` method. The instance for my problem
can be represented by the first two terms of the sequence, and the third
term, which is the answer. I choose to use the limits to represent the lowest
value, and the largest value for the first two terms of the sequence.

```python
def genfun(cls, limits):
    min_val, max_val = limits
    vals = list(range(min_val, max_val + 1))
    a1, a2 = rnd.choice(vals), rnd.choice(vals)

    difference = a2 - a1
    a3 = a2 + difference
    return a1, a2, a3
```

In the last step of the method we return the instance (which is composed of the
first three terms in this case).

Now, we implement `testfun` to make sure our instances are valid.

```python
def testfun(cls, instance):
    a1, a2, a3 = instance  # "unpack" the instance
    return a2 - a1 == a3 - a2
```

Finally, we implement `create_question` to transform instances
into questions.

```python
def create_question(cls, instance, tol, points=[1]):
    a1, a2, a3 = instance
    return f"""
    Find the third term of an arithmetic sequence, if the first terms 
    are: {a1}, {a2}. 
    
    <m>a_3 = </m> :n
    :n={Task.answer(a3, 4, tol)} {points[0]} ok
    """
```

This will create a question for a given instance, if we also set the
tolerance `tol` for the answer (in this case it could easily be `0`, typically `0.0001`).
In this case, the student will receive 1 point for a correct answer (within the given tolerance).

Now, I will separate the code generating the answer from the `genfun` method and
"hide" it inside the `solve_fun` method, so it can be easily removed later in the final version.

In the end, the file `arithmetic_sequence.py` will look like this:

```python
import random as rnd

from task import Task

class ArithmeticSequence(Task):

    @classmethod
    def create_question(cls, instance, tol, points=[1]):
        a1, a2, a3 = instance
        return f"""
        Find the third term of an arithmetic sequence, if the first terms 
        are: {a1}, {a2}. 
        
        <m>a_3 = </m> :n
        :n={Task.answer(a3, 4, tol)} {points[0]} ok
        """
    
    @staticmethod
    def solve_instance(a1, a2):
        difference = a2 - a1
        a3 = a2 + difference
        return a3
    
    @classmethod
    def genfun(cls, limits):
        min_val, max_val = limits
        vals = list(range(min_val, max_val + 1))
        a1, a2 = rnd.choice(vals), rnd.choice(vals)
    
        a3 = cls.solve_instance(a1, a2)
        return a1, a2, a3
    
    @classmethod
    def testfun(cls, instance):
        a1, a2, a3 = instance  # "unpack" the instance
        return a2 - a1 == a3 - a2
```

### Using the new class

From the Python CLI, I will load the class `ArithmeticSequence` and test it.
```python
>>> ArithmeticSequence.test(1000)
0 errors in 1000 instances
```

Because there seem to be no errors, I proceed to generate 20 questions with it.
Before that, I need to create a new folder `questions` inside the `L0` folder.
```python
>>> ArithmeticSequence.questions()
```

I can check the outcome inside the file `questions/ArithmeticSequence.txt`, which
can be directly passed into a ROPOT.

\end{markdown*}
\shorthandon{-}








\chapter{Glossary}

All these keywords are explained in the text of the thesis when they are first introduced. However, it might be difficult to look for them if you do not read the text sequentially.


\paragraph{instance}

A collection of data that uniquely represents a given question, including the answer to it. E.g., for the question ``\textit{Compute the sum $1 + 3$}'', the instance could be \textit{1, 3, 4.}

\paragraph{IS}

The information system for MUNI, available at \href{https://is.muni.cz/}{https://is.muni.cz/}.

\paragraph{question}

A human-readable text containing the problem formulation and all the necessary data to solve it. It has a unique answer. E.g., the text ``\textit{Compute the sum $1 + 3$}.'' is a question.


\paragraph{question set}

This is usually a file containing one or more questions. An example can be found inside the root of the project: \newline \verb|sample_question_set.txt|.

\paragraph{ROPOT}

\textbf{R}evision \textbf{O}pinion \textbf{PO}ll and \textbf{T}esting is an online application in IS capable of assigning problems to students and evaluating their answers. It can parse question sets.


\paragraph{task}

The implementation for generation of a concrete problem. E.g., the class \verb|ImplicitLineFromTwoPoints|.


\appendix

\chapter{Implemented tasks}

\shorthandoff{-}
\begin{markdown*}{%
  hybrid,
  definitionLists,
  footnotes,
  inlineFootnotes,
  hashEnumerators,
  fencedCode,
  citations,
  citationNbsps,
  pipeTables,
  tableCaptions,
}

In this chapter, I will focus on listing all the implemented tasks while providing a high-level commentary for each of them. Instead of dwelling on implementation details, I will show the results and explain what knowledge the tasks aim to test.

Each section corresponds to a lecture and each subsection corresponds to a task for the given lecture.

\section{Lecture 1}

The goal of this lecture is to revise high school concepts, the basics of analytical geometry. It covers

* points, vectors, norm, and dot product;
* barycentric coordinates;
* parametric, implicit equations of 2D lines;
* point to line distance;
* intersection of two lines.

\subsection{AngleBetweenVectors}

This task makes sure the student understands the connection between the dot product of two vectors and the size of the angle between them.

\subsubsection{Problem formulation}
Given two 3D vectors $u$ and $v$, find the angle between them.

\subsubsection{ROPOT question example}

```
Find the angle <m>\theta \in [0, \pi]</M> between vectors 
<m>u</m> and <m>v</m> where <m>u = (-2,4,3)^\top, \; 
v = (0,-2,3)^\top </m>. Write the answer in radians 
(if the answer is a floating point number round it to 
4 decimal places).

<m>\theta = </M> :n

:n=" 1.5192.. 1.5194" 2 ok  
```

\subsection{DistancePointFromImplicitLine}

By completing this task, the student shows the knowledge and understanding of the point--line distance formula. 

\subsubsection{Problem formulation}
Find the distance between a given point $P$ and a line $l$, where $l$ is a 2D line given implicitly as $ax + by + c = 0$.

\subsubsection{ROPOT question example}

```
Find the distance <m>d</m> from the line <m>l: -4x +2y -2 = 0</m> 
to the point <m>[5,-5]^\top</m>

<m>d = </m> :n

:n=" 7.1553.. 7.1555" 2 ok
```

\subsection{DistancePointFromParametricLine}

In order to solve this task, the student must understand basic notions in 2D geometry, especially direction and normal line vectors, and the parametric equation of line.

\subsubsection{Problem formulation}
Find the distance between the point $P$ and the line $l$, where $l$ is a 2D line given parametrically as $l(t) = Q + t\vec v$.

\subsubsection{ROPOT question example}

```
Find the distance <m>d</m> from the line <m> l(t) = 
\begin{bmatrix} 5 \\ -2 \end{bmatrix} + t 
\begin{pmatrix} -5 \\ 0 \end{pmatrix} </m> to 
the point <m>[-4,-4]^\top</m>

<m>d = </m> :n

:n=" 1.9999.. 2.0001" 2 ok
```

\subsection{EquallyDistantPointOnImplicitLine}

This task ensures the student has a deeper understanding of 2D geometry. The student should be able to form his thoughts as equations and solve them.   

\subsubsection{Problem formulation}
Find the point $Q$ on the line $ax + by + c = 0$, which is equally distant to the given points $P$ and $R$.

\subsubsection{ROPOT question example}

```
Find the point on the line <m>- x + 3 y + 5 = 0</m> which is 
equally distant from the points <m>[9, -2]^\top</m> and 
<m>[1, -8]^\top</m> (if the answer is a floating point 
number round it to 4 decimal places).
<m>x =</m> :n1
<m>y =</m> :n2

:n1=" 1.9999.. 2.0001" 2 ok
:n2="-1.0001..-0.9999" 2 ok
```


\subsection{ImplicitEquationFromParametric}

To solve this task, the student must understand basic equations of 2D line.

\subsubsection{Problem formulation}
Given the parametric equation of a 2D line, find its implicit equation.

\subsubsection{ROPOT question example}

```
Fill in the missing values of the coefficients 
<m>a, b</m> and <m>c</m> of the implicit equation of 
the line <m>l(t) = \begin{bmatrix} 3 \\ 3 \end{bmatrix} + 
t\begin{pmatrix} 3 \\ 5 \end{pmatrix}</m> (if the answer 
is a floating point number round it to 4 decimal places).
 
<m>a= </m> :n1 </br>
<m>b = 5 </m> </br>
<m>c = </m> :n2 </br>

:n1="-8.3334..-8.3332" 2 ok
:n2=" 9.9999.. 10.0001" 2 ok
```

\subsection{ImplicitLineFromPointAndParallel}

The student should have a basic notion of the implicit equation of line and how it is formed.

\subsubsection{Problem formulation}
Find the implicit equation of a line passing through the point $P$ and parallel to the vector $\vec v$ (or $y$-axis), given the coefficient $a$.

\subsubsection{ROPOT question example}

```
Fill in the missing values of the coefficients <m>a, 
b</m> and <m>c</m> of the implicit equation of the line 
which goes through the point <m>[-1,-4]^\top</m> 
parallel to the vector <m>(2,-4)^\top</m> 
(if the answer is a floating point number round it to 
4 decimal places).

<M>a = 1</m>
<M>b = </m> :n2
<M>c = </m> :n3

:n2=" 0.4999.. 0.5001" 2 ok
:n3=" 2.9999.. 3.0001" 2 ok
```

\subsection{ImplicitLineFromTwoPoints}

The student must have a deeper understanding of the implicit line equation and vectors in general. 

\subsubsection{Problem formulation}
Find the implicit equation of a line passing through the points $P$ and $Q$, given the coefficient $b$.

\subsubsection{ROPOT question example}

```
Fill in the missing values of the coefficients <m>a, 
b</m> and <m>c</m> of the implicit equation of 
the line which goes through the points <m>[-1,-2]^\top</m> 
and <m>[4,3]^\top</m> (if the answer is a floating 
point number round it to 4 decimal places).

<M>a = </m> :n1
<M>b = -5</m>
<M>c = </m> :n3

:n1=" 4.9999.. 5.0001"  2 ok
:n3="-5.0001..-4.9999"  2 ok
```

\subsection{ImplicitLineThroughTriangle}

This task ensures the student is able to apply the learned techniques to more general problems.

\subsubsection{Problem formulation}
Find the implicit equation $ax + by + c = 0$ of the line passing through $A$ and perpendicular to $BC$, given vertices of the triangle $ABC$ and coefficient $a$.

\subsubsection{ROPOT question example}

```
The vertices of the triangle <m>ABC</m> are given 
<m>A=[4,-2]^\top, B=[2,-1]^\top</m> and <m>C=[5,-5]^\top</m>. 
Write the coefficients <m>a, b</m> and <m>c</m> of 
the implicit equation of the line which is perpendicular to 
<m>BC</m> and goes through the point <m>A</m>.

<m>a = -1 </m> 
<m>b = </m> :n2
<m>c = </m> :n3

:n2=" 1.3332.. 1.3334" 2 ok
:n3=" 6.6666.. 6.6668" 2 ok
```

\subsection{ParametricEquationFromImplicit}

To solve this task, the student must understand basic equations of 2D line.

\subsubsection{Problem formulation}
Given the implicit equation of a 2D line, find its parametric equation.

\subsubsection{ROPOT question example}

```
Fill in the missing values of <m>P = [p_1, p_2]^\top</m> and 
<m>v = (v_1, v_2)^\top</m> in the parametric equation of 
the following line <m>l: - x - y - 2 = 0</m> (if the answer 
is a floating point number round it to 4 decimal places).

<m>p_1 = 0</m></br>
<m>p_2 = </m> :n1 </br>
<m>v_1 = 3</m></br>
<m>v_2 = </m> :n2 </br>
        
:n1="-2.0001..-1.9999" 2 ok 
:n2="-3.0001..-2.9999" 2 ok
```

\subsection{ParametricEquationFromPointAndVector}

In order to solve this task, the student should understand the basics of the parametric line equation and how it is formed.

\subsubsection{Problem formulation}
Find the parametric equation of a line $l(t) = P + t\vec v$ passing through the point $Q$ and parallel to the vector $\vec u$, given the x-coordinate of $P$ and the y-coordinate of $\vec v$.

\subsubsection{ROPOT question example}

```
Fill in the values of <m>P=[p_1,p_2]^\top</m> and 
<m>v=(v_1,v_2)^\top</m> in the parametric equation of 
the line which goes through the point <m>[-1,-5]^\top</m> 
parallel to the vector <m>(-1,-4)^\top</m> (if the answer 
is a floating point number round it to 4 decimal places).

<m>p_1 = 2 </m> 
<m>p_2 =  </m> :n2
<m>v_1 =  </m> :n3
<m>v_2 = 1 </m> 

:n2=" 6.9999.. 7.0001" 2 ok
:n3=" 0.2499.. 0.2501" 2 ok
```


\subsection{ParametricEquationFromSlopeAndPoint}

By completing this task, the student proves a deeper understanding of the parametric line equation.

\subsubsection{Problem formulation}
Find the parametric equation of a line $l(t) = P + t\vec v$ with a given slope and passing through the point $Q$, given y-coordinates of $P$ and $\vec v$.

\subsubsection{ROPOT question example}

```
Fill in the values of <m>P=[p_1,p_2]^\top</m> and 
<m>v=(v_1,v_2)^\top</m> in the parametric equation of 
the line which goes through the point <m>[2,-3]^\top</m> and 
has the slope (in the explicit equation) <m>\hat{a} = -2</m> 
(if the answer is a floating point number round it to 
4 decimal places).

<m>p_1 = </m> :n1
<m>p_2 = 1</m>
<m>v_1 = </m> :n3
<m>v_2 = 1</m>

:n1="-0.0001.. 0.0001" 2 ok
:n3="-0.5001..-0.4999" 2 ok
```

\subsection{PointProjectionOnImplicitLine}

This task tests comprehension of the basic concepts of analytic geometry and basic problem solving skills. 

\subsubsection{Problem formulation}
Find the projection $Q$ of the given point $P$ on the line $ax + by + c = 0$.

\subsubsection{ROPOT question example}

```
Find the projection of the point <m>[4,-2]^\top</m> 
on the line <m>3 y = 0</m>.

<m>x =</m> :n1
<m>y =</m> :n2

:n1=" 3.9999.. 4.0001" 2 ok
:n2="-0.0001.. 0.0001" 2 ok
```

\subsection{ScalarProductFromAngle}

This task ensures the student has a notion of the connection between scalar product of two vectors and  the angle between them.

\subsubsection{Problem formulation}
Given norms of vectors $u$ and $v$ and the angle between them, find the scalar product $u\cdot v$.

\subsubsection{ROPOT question example}

```
Find the scalar product of vectors <m>u</m> and 
<m>v</m> where <m>\|u\| = 3, \; \|v\| = 1, </m> and 
<m>u \perp v </m> (if the answer is a floating point 
number round it to 4 decimal places).

<M>u \cdot v = </m> :n

:n="-0.0001.. 0.0001" 2 ok   
```

\subsection{ScalarProductFromVectors}

This task tests students' basic understanding of scalar product. 

\subsubsection{Problem formulation}
Given coordinates of 3D vectors $u$ and $v$, find the scalar product $u\cdot v$.

\subsubsection{ROPOT question example}

```
Find the scalar product of vectors <m>u</m> and 
<m>v</m> where <m>u = (-3,3,-5)^\top, \; 
v = (5,-5,-3)^\top </m> (if the answer is 
a floating point number round it to 4 decimal places).

<m>u \cdot v = </m> :n

:n="-15.0001..-14.9999" 2 ok     
```

\subsection{SymmetricPointThroughImplicitLine}

This task ensures the student has a deeper understanding of 2D geometry. The student should be able to form his thoughts as equations and solve them. 

\subsubsection{Problem formulation}
Find the point $Q$ which is symmetric to the given point $P$ with respect to the given line $ax + by + c = 0$.

\subsubsection{ROPOT question example}

```
Find the point which is symmetric to the point 
<m>[-5,3]^\top</m> with respect to the line 
<m>2 x - 2 y - 1 = 0</m>.

<m>x =</m> :n1
<m>y =</m> :n2

:n1=" 3.4999.. 3.5001" 2 ok
:n2="-5.5001..-5.4999" 2 ok 
```

\section{Lecture 2}

This lecture covers linear maps in 2D, their composition and homogeneous coordinates. 
Important linear maps are:

* scaling,
* reflection,
* rotation,
* shear,
* and projection.

The properties of linear maps (such as area change) are explained, as well as inverse matrices. 

\subsection{DeterminantOfTransformationMatrix}

In order to solve this task, the student must know the geometrical meaning of transformation determinant. It also forces the student to deal with square roots in the matrix values.    

\subsubsection{Problem formulation}
Find the ratio of change of a given area through the transformation given by a $2\times2$ transformation matrix.

\subsubsection{ROPOT question example}

```
Let <m>ABC</m> be a triangle, where <m>A=[1,-5]^\top</m>, 
<m>B=[5,-2]^\top</m>, and <m>C=[-1,-3]^\top</m>. 
How will the transformation <m>M=\begin{pmatrix} 
\frac{1}{2}+\frac{\sqrt{3}}{2} & -\frac{\sqrt{3}}{1} \\ 
\frac{7}{4} & -5  \end{pmatrix}</m> change the value of 
product <m>\|{BH}\| \cdot \|{AC}\|</m>, 
where <m>H\in AC</m> and <m>BH\bot AC</m>? 
Enter the ratio of the change (the answer equal to 1 
means no change, the answer less than 1 means it makes it smaller, 
and the answer bigger than 1 means it makes it larger). 

Ratio = :n1

:n1="-3.7991..-3.7989" 4 ok
```

\subsection{HomogeneousMatrixOfComposition}

The student should be able to understand the connection between a transformation and its corresponding homogeneous matrix and how translation composition translates into matrix product. This task is a homogeneous equivalent of the `MatrixOfComposition` task.

\subsubsection{Problem formulation}
Find the homogeneous matrix of composition of 3 homogeneous transformations.

\subsubsection{ROPOT question example}

```
Fill in the coefficients of the transformation matrix <m>M</m> 
in homogeneous coordinates which performs the following transform: 
scale by 2 in <m>x</m>-coordinate and by 2 in <m>y</m>-coordinate, 
rotation for <m>\frac{2}{3}\pi</m> radians around <m>z</m>-axis, 
and translation by the vector (1, 5, 3).

If the answer is a floating point number round it to 
4 decimal places.

<m>M_{11} = </m> :n11 <m>M_{12} = </m> :n12 
<m>M_{13} = </m> :n13 <m>M_{14} = </m> :n14

<m>M_{21} = </m> :n21 <m>M_{22} = </m> :n22 
<m>M_{23} = </m> :n23 <m>M_{24} = </m> :n24

<m>M_{31} = </m> :n31 <m>M_{32} = </m> :n32 
<m>M_{33} = </m> :n33 <m>M_{34} = </m> :n34

<m>M_{41} = </m> :n41 <m>M_{42} = </m> :n42 
<m>M_{43} = </m> :n43 <m>M_{44} = </m> :n44

:n11="-1.0001..-0.9999" 1 ok
:n12="-1.7322..-1.7320" 1 ok
:n13="-0.0001.. 0.0001" 1 ok
:n14=" 0.9999.. 1.0001" 1 ok

:n21=" 1.7320.. 1.7322" 1 ok
:n22="-1.0001..-0.9999" 1 ok
:n23="-0.0001.. 0.0001" 1 ok
:n24=" 4.9999.. 5.0001" 1 ok

:n31="-0.0001.. 0.0001" 1 ok
:n32="-0.0001.. 0.0001" 1 ok
:n33=" 0.9999.. 1.0001" 1 ok
:n34=" 2.9999.. 3.0001" 1 ok

:n41="-0.0001.. 0.0001" 0 ok
:n42="-0.0001.. 0.0001" 0 ok
:n43="-0.0001.. 0.0001" 0 ok
:n44=" 0.9999.. 1.0001" 0 ok
```

\subsection{InverseMatrix}

This task ensures the student is able to compute the inverse of a $2\times 2$ matrix.

\subsubsection{Problem formulation}
Find the inverse matrix for a given $2\times 2$ matrix.

\subsubsection{ROPOT question example}

```
Compute the inverse matrix <m>A^{-1}</m> of the matrix 
<m>A=\begin{pmatrix} -3 & -1 \\ -4 & -2 \end{pmatrix}</m> 
(if the answer is a floating point number 
round it to 4 decimal places).

<m>a_{11} = </m> :n1 <m>a_{12} = </m> :n2
<m>a_{21} = </m> :n3 <m>a_{22} = </m> :n4

:n1="-1.0001..-0.9999" 1 ok
:n2=" 0.4999.. 0.5001" 1 ok
:n3=" 1.9999.. 2.0001" 1 ok
:n4="-1.5001..-1.4999" 1 ok
```

\subsection{MatrixOfComposition}

The student should be able to understand the connection between a transformation and its corresponding matrix and how translation composition translates into matrix product.     

\subsubsection{Problem formulation}
Find the matrix of composition of 3 transformations.

\subsubsection{ROPOT question example}

```
Fill in the matrix of sequence of the 
following three transformations: 
(1) projection on <m>x</m>-axis, 
(2) reflection with respect to <m>y=x</m> line, 
and then (3) shear for 5 in <m>x</m>-coordinate
(if the answer is a floating point number 
round it to 4 decimal places).

<m>a_{11} = </m> :n1 <m>a_{12} = </m> :n2
<m>a_{21} = </m> :n3 <m>a_{22} = </m> :n4

:n1=" 4.9999.. 5.0001" 1 ok
:n2="-0.0001.. 0.0001" 1 ok
:n3=" 0.9999.. 1.0001" 1 ok
:n4="-0.0001.. 0.0001" 1 ok
```

\subsection{MatrixOfGivenTransformation}

This task tests the student understands the matrices of basic transformations.

\subsubsection{Problem formulation}

Find the matrix corresponding to a given transformation.

\subsubsection{ROPOT question example}

```
Fill in the matrix of rotation for 
<m>-\frac{1}{4}\pi</m> radians (if the answer is 
a floating point number round it to 4 decimal places).

<m>a_{11} = </m> :n1 <m>a_{12} = </m> :n2
<m>a_{21} = </m> :n3 <m>a_{22} = </m> :n4

:n1=" 0.7070.. 0.7072" 1 ok
:n2=" 0.7070.. 0.7072" 1 ok
:n3="-0.7072..-0.7070" 1 ok
:n4=" 0.7070.. 0.7072" 1 ok
```

\subsection{MatrixPreservesNormsAndAngles}

The student can easily solve this task if he understands the geometric meaning of matrices as transformations. Another option is solve it the hard way checking all given vectors and angles between them.

\subsubsection{Problem formulation}

Check if the given $2\times2$ matrix preserves norms and angles of given vectors.

\subsubsection{ROPOT question example}

```
Check if the transformation <m>A=\begin{pmatrix} 
0 & \frac{\sqrt{3}}{2} \\ - \frac{\sqrt{3}}{2} & 0 
\end{pmatrix}</m> preserves the norms of the following 
vectors  and the angles between each pair of them: 
<m>a = (3, 1)^\top</m>, <m>b = (4, 1)^\top</m>, 
<m>c = (1, 1)^\top</m>, <m>d = (1, 0)^\top</m>. 
Answer "y" if it preserves the norms of ALL vectors given, 
"n" if it doesn't preserve the norm of ANY of them.

Does it preserve the norms? (y/n) :t1
Does it preserve the angles? (y/n) :t2

:t1="n" 2 ok
:t2="y" 2 ok
```

\subsection{RotationMatrixFromVectors}

This task makes sure the students understand the composition of scale and rotation matrices and their effect on vectors.

\subsubsection{Problem formulation}

Find the $2\times2$ matrix of rotation given an input and its corresponding output vector.

\subsubsection{ROPOT question example}

```
Find matrix <m>A</m> if <m>v'=BAv</m>. 
<m>A</m> is a matrix of rotation, 
<m>B</m> scales <m>x</m> and <m>y</m> by 
the same ratio and <m>v=(2, -4)^\top;v'=(2, 4)^\top</m>. 
<m>a_{11}</m> is a negative number.
If the answer is a floating point number 
round it to 4 decimal places.

<m>a_{11} = </m> :n1 <m>a_{12} = </m> :n2
<m>a_{21} = </m> :n3 <m>a_{22} = </m> :n4

:n1="-0.6001..-0.5999" 1 ok
:n2="-0.8001..-0.7999" 1 ok
:n3=" 0.7999.. 0.8001" 1 ok
:n4="-0.6001..-0.5999" 1 ok
```

\section{Lecture 3}

This lecture focuses on matrix eigenvectors and eigenvalues. It explains their definition, computation (even numerical methods), and geometrical meaning. It covers the properties of symmetric matrices and matrix diagonalization.

\subsection{Deflation}

To solve this task, the student must know how to find the Hotteling's deflation.

\subsubsection{Problem formulation}

Find the Hotteling's deflation of a $3\times3$ matrix for a given eigenvalue and eigenvector of the matrix.

\subsubsection{ROPOT question example}

```
Fill in the values of the matrix <m>A'</m> which is 
the result of Hotteling's deflation for the matrix 
<m>\left[\begin{matrix}0 & 0 & 0\\
2 & 3 & -1\\ 0 & -5 & -1\end{matrix}\right]</m>, 
eigenvalue <m>\lambda_1=4</m> and corresponding eigenvector 
<m>u_1=[0,- \frac{\sqrt{2}}{2},\frac{\sqrt{2}}{2}]^\top</m>.
If the answer is a floating point number 
round it to 4 decimal places.

<m>a'_{11}</m> = :n1 <m>a'_{12}</m> = :n2 <m>a'_{13}</m> = :n3
<m>a'_{21}</m> = :n4 <m>a'_{22}</m> = :n5 <m>a'_{23}</m> = :n6
<m>a'_{31}</m> = :n7 <m>a'_{32}</m> = :n8 <m>a'_{33}</m> = :n9

:n1="-0.0001.. 0.0001" 1 ok
:n2="-0.0001.. 0.0001" 1 ok
:n3="-0.0001.. 0.0001" 1 ok

:n4=" 1.9999.. 2.0001" 1 ok
:n5=" 0.9999.. 1.0001" 1 ok
:n6=" 0.9999.. 1.0001" 1 ok

:n7="-0.0001.. 0.0001" 1 ok
:n8="-3.0001..-2.9999" 1 ok
:n9="-3.0001..-2.9999" 1 ok
```

\subsection{EigenvaluesEigenvectors2x2}

This task ensures the student is able to compute the eigenvectors and eigenvalues for the simplest
case -- $2\times2$ matrix.

\subsubsection{Problem formulation}

Find the eigenvalues and corresponding eigenvectors for a given $2\times2$ matrix.

\subsubsection{ROPOT question example}

```
Find the eigenvalues and eigenvectors of matrix 
<m>A=\begin{pmatrix} -1 & 0 \\ 1 & 0 \end{pmatrix}</m>.
The eigenvalues should be in descending order 
(<m>\lambda_1 > \lambda_2</m>). The eigenvectors should be 
normalized <m>\|u_{i}\|=1</m> and have positive 
first component <m>u_{i1} > 0</m>. If the answer is 
a floating point number round it to 4 decimal places.

<m>\lambda_1</m> = :n1
<m>\lambda_2</m> = :n2
<m>u_{11}</m> = :n3
<m>u_{12}</m> = :n4
<m>u_{21}</m> = :n5
<m>u_{22}</m> = :n6

:n1="-0.0001.. 0.0001" 1 ok
:n2="-1.0001..-0.9999" 1 ok
:n3="-0.0001.. 0.0001" 1 ok
:n4=" 0.9999.. 1.0001" 1 ok
:n5=" 0.7070.. 0.7072" 1 ok
:n6="-0.7072..-0.7070" 1 ok
```

\subsection{EigenvaluesEigenvectors3x3}

This task ensures the student is able to compute the eigenvectors and eigenvalues for $3\times3$ matrices. And thus, hopefully, in the general case as well.

\subsubsection{Problem formulation}

Find the eigenvalues and corresponding eigenvectors for a given $3\times3$ matrix.

\subsubsection{ROPOT question example}

```
Find the eigenvalues and eigenvectors of matrix <m>A=
\begin{pmatrix} 1 & 0 & 5 \\ 0 & -5 & 1 \\ 0 & 3 & -3 
\end{pmatrix}</m>. The eigenvalues should be in descending 
order (<m>\lambda_1 > \lambda_2 > \lambda_3</m>). 
The eigenvectors should be normalized <m>\|u_{i}\|=1</m> 
and have positive first nonzero component <m>u_{iJ} > 0 </m> 
where <m>u_{ij} = 0 </m> for <m>j=1,\dots,J-1</m>.
If the answer is a floating point number round it to 
4 decimal places. If there are less than 3 unique 
eigenvalues enter zeros in the corresponding fields.

<m>\lambda_1</m> = :n1
<m>\lambda_2</m> = :n2
<m>\lambda_3</m> = :n3
<m>u_{11}</m> = :n4
<m>u_{12}</m> = :n5
<m>u_{13}</m> = :n6
<m>u_{21}</m> = :n7
<m>u_{22}</m> = :n8
<m>u_{23}</m> = :n9
<m>u_{31}</m> = :n10
<m>u_{32}</m> = :n11
<m>u_{33}</m> = :n12

:n1=" 0.9999.. 1.0001" 1 ok
:n2="-2.0001..-1.9999" 1 ok
:n3="-6.0001..-5.9999" 1 ok

:n4=" 0.9999.. 1.0001" 1 ok
:n5="-0.0001.. 0.0001" 1 ok
:n6="-0.0001.. 0.0001" 1 ok

:n7=" 0.8451.. 0.8453" 1 ok
:n8="-0.1691..-0.1689" 1 ok
:n9="-0.5072..-0.5070" 1 ok

:n10=" 0.4507.. 0.4509" 1 ok
:n11=" 0.6311.. 0.6313" 1 ok
:n12="-0.6313..-0.6311" 1 ok
```

\subsection{MatrixFromEigenvectorsEigenvalues}

In order to solve this task, the student needs to know the definition of eingevectors and eigenvalues and use basic algebraic reasoning. It also ensures the student understands what symmetric matrices are.

\subsubsection{Problem formulation}

Find the symmetric $2\times2$ matrix given its eigenvectors and eigenvalues.

\subsubsection{ROPOT question example}

```
Construct a symmetric matrix A of type 2x2 having 
the given (orthonormal) eigenvectors <m>u_1</m>, <m>u_2</m> 
and eigenvalues <m>\lambda_1</m>, <m>\lambda_2</m> if:
<m>u_1=\frac{1}{\sqrt{4}} \begin{pmatrix} 0 \\
-2 \end{pmatrix}</m>; <m>u_2=\frac{1}{\sqrt{4}} 
\begin{pmatrix} -2 \\ 0 \end{pmatrix}</m> 
<m>\lambda_1=-2</m>; <m>\lambda_2=-4</m> If the answer 
is a floating point number round it to 4 decimal places.

<m>a_{11}</m> = :n1
<m>a_{12}</m> = :n2
<m>a_{21}</m> = :n3
<m>a_{22}</m> = :n4

:n1="-4.0001..-3.9999" 1 ok
:n2="-0.0001.. 0.0001" 1 ok
:n3="-0.0001.. 0.0001" 1 ok
:n4="-2.0001..-1.9999" 1 ok
```

\subsection{PowerMethod}

In order to solve this task, the student must be able to follow the Power Method Algorithm.

\subsubsection{Problem formulation}

Compute the third iteration of power method for a given initial vector and a $3\times3$ matrix.

\subsubsection{ROPOT question example}

```
Write the estimates of the eigenvalue <m>\lambda_1</m> and 
eigenvector (normalized) <m>u_1</m> of the matrix 
<m>A=\begin{pmatrix} 0 & 0 & -1 \\ -5 & -1 & 3 \\ 
-2 & 2 & -1 \end{pmatrix}</m> after 3 iterations of power
method if the initial vector is <m>x^0=[1, -2, 5]^\top</m>.
Eigenvalue <m>\lambda_1</m> is the largest eigenvalue 
in absolute value of the given matrix. The corresponding 
eigenvector <m>u_1</m> has positive first nonzero component 
<m>u_{iJ} > 0 </m> where <m>u_{ij} = 0 </m> for 
<m>j=1,\dots,J-1</m>. If the answer is a floating point 
number round it to 4 decimal places.

<m>\lambda_1</m> = :n1
<m>u_{11}</m> = :n2
<m>u_{12}</m> = :n3
<m>u_{13}</m> = :n4

:n1=" 2.3777.. 2.3779" 1 ok
:n2="-0.2938..-0.2936" 1 ok
:n3=" 0.6526.. 0.6528" 1 ok
:n4="-0.6985..-0.6983" 1 ok
```

\subsection{VectorInEigenvectorBasis}

This task tries to stress that eigenvectors of a matrix form a basis. The student should be able to translate vectors between different bases. 

\subsubsection{Problem formulation}

Find the coordinates of a given vector in the basis of eigenvectors of a given $2\times2$ matrix.

\subsubsection{ROPOT question example}

```
Find the coordinates of the vector <m>v=\begin{pmatrix} 
-2 \\ 1 \end{pmatrix}</m> in the basis of eigenvectors of 
the matrix <m>A=\begin{pmatrix} -3 & -3 \\ -3 & -3 
\end{pmatrix}</m>. The eigenvalues should be in descending 
order (<m>\lambda_1 > \lambda_2</m>). The eigenvectors 
should be normalized <m>\|u_{i}\|=1</m> and have 
positive first component <m>u_{i1} > 0</m>. If the answer 
is a floating point number round it to 4 decimal places.

<m>v = \begin{pmatrix} v_1 \\ v_2 \end{pmatrix}</m> = 
:n1 <m>u_{1}</m> + :n2 <m>u_{2}</m>

:n1="-2.1214..-2.1212" 2 ok
:n2="-0.7072..-0.7070" 2 ok
```

\section{Lecture 4}

*This lecture was left out intentionally. The lecturer will implement the question sets to try out the project functionality.* 
 \newline

The general topic of the lecture is Principal Component Analysis. 

\section{Lecture 5}

This lecture covers geometric objects interactions, such as relative position in space and collision detection. It revises the parametric and implicit equations of lines and planes (and barycentric coordinates).

It covers the following interactions:

* line--line,
* line--plane, 
* line--plane.

It analyzes in detail these types of collision:

* sphere--plane,
* sphere--sphere, 
* and box--plane.


\subsection{BarycentricCoordinates}

This task ensures the student knows how barycentric coordinates work.

\subsubsection{Problem formulation}

Find the barycentric coordinates of a given 3D point with respect to 3 given points.

\subsubsection{ROPOT question example}

```
Find the barycentric coordinates <m>u_1,u_2,u_3</m> of 
the point <m>X=[0,-8,-32]^\top</m> with respect to the points 
<m>P_1=[2,1,4]^\top</m>, <m>P_2=[-3,0,1]^\top</m> and 
<m>P_3=[-2,1,4]^\top</m>.

<m>X=u_1P_1+u_2P_2+u_3P_3</m>
<m>u_1 = </m> :n1
<m>u_2 = </m> :n2
<m>u_3 = </m> :n3

:n1="-4.0001..-3.9999" 2 ok
:n2="-0.0001.. 0.0001" 2 ok
:n3="-4.0001..-3.9999" 2 ok

```

\subsection{CollisionSpherePlane}

In order to solve this task, the student must understand the theory behind sphere--plane collision.

\subsubsection{Problem formulation}

Find the time and point of collision of an implicit plane and a sphere $(P, r)$ travelling from point $P$ with a constant vector of speed $\vec v$.

\subsubsection{ROPOT question example}

```
Given a plane <m>8 - 4 z=0</m> and a sphere <m>S(P,\rho)</m>, 
where <m>\rho=5</m> meters moving from the point 
<m>P_1 = [5.0,-1.0,-19.0]^\top</m> at time <m>t=0</m>s 
with the constant velocity <m>v = (0, -1, 4)^\top</m> m/s. 
Decide if the sphere can collide with the plane. 
If yes, fill the time <m>\hat{t}</m> at which the collision 
happens and enter the coordinates of the touching point 
<m> C </m>. If no, fill in "x" to all the fields.
If the answer is a floating point number 
round it to 4 decimal places.

<m> \hat{t} = </m> :n1 seconds
<m> C_1 = </m> :n2 meters
<m> C_2 = </m> :n3 meters
<m> C_3 = </m> :n4 meters

:n1=" 3.9999.. 4.0001" 2 ok
:n2=" 4.9999.. 5.0001" 1 ok
:n3="-5.0001..-4.9999" 1 ok
:n4=" 1.9999.. 2.0001" 1 ok
```

\subsection{CollisionSphereSphere}

In order to solve this task, the student must understand the theory behind sphere--sphere collision.

\subsubsection{Problem formulation}

Find the time and point of collision of two spheres travelling at constant velocity during 1 second from a given point to a given point.

\subsubsection{ROPOT question example}

```
There are two spheres <m>S_1(P,\rho_P)</m> and 
<m>S_2(Q,\rho_Q)</m> moving in the space, where 
<m>P_1 = [1,2,-1]^\top, \rho_P=3</m>, 
<m>Q_1=[-1,5,-1]^\top, \rho_Q=4</m>. The sphere 
<m>S_1</m> travels to the point <m>P_2=[-3,-1,-2]^\top</m> 
and the sphere <m>S_2</m> travels to the point 
<m>Q_2=[-5,-3,4]^\top</m> with constant velocity during 1 second.
Decide if the spheres will collide. If yes, fill the time 
<m>\hat{t}</m> at which the collision happens and enter 
the coordinates of the touching point <m> C </m>. 
If no, fill in "x" to all the fields. If the answer is 
a floating point number round it to 4 decimal places.

<m> \hat{t} = </m> :t1 seconds
<m> C_1 = </m> :t2 meters
<m> C_2 = </m> :t3 meters
<m> C_3 = </m> :t4 meters

:t1="x" 2 ok
:t2="x" 1 ok
:t3="x" 1 ok
:t4="x" 1 ok
```

\subsection{CrossProductFromAngle}

This task ensures the student has a notion of the connection between cross product of two vectors and the angle between them. The knowledge of special cases (parallel vectors) is also useful to speed up the computation.

\subsubsection{Problem formulation}

Given norms of vectors $\vec u$ and $\vec v$ and the angle between them, find the cross product $\vec u\times \vec v$.

\subsubsection{ROPOT question example}

```
Find the cross product of vectors <m>u</m> and <m>v</m> where 
<m>\|u\| = 3, \; \|v\| = 4, </m> and 
<m>u \uparrow\downarrow v</m> (if the answer is a floating 
point number round it to 4 decimal places).

<M>\|{a} \times {b}\| = </m> :n

:n="-0.0001.. 0.0001" 2 ok     
```

\subsection{CrossProductFromVectors}

In order to solve this task, the student must be able to apply the basic definition of the cross product. The knowledge of special cases (parallel vectors) is also useful to speed up the computation.

\subsubsection{Problem formulation}

Given vectors $\vec u$ and $\vec v$, find the cross product $\vec u \times \vec v$.

\subsubsection{ROPOT question example}

```
Find the cross product of vectors <m>{u}</m> and 
<m>{v}</m> where <m>{u} = (-2,-2,2)^\top, \; 
{v} = (5,-2,-5)^\top </m> (if the answer is 
a floating point number round it to 4 decimal places).

<M>{c}={u} \times {v}</m>
<M>c_1 = </m> :n1
<M>c_2 = </m> :n2
<M>c_3 = </m> :n3

:n1=" 13.9999.. 14.0001" 2 ok
:n2="-0.0001.. 0.0001" 2 ok
:n3=" 13.9999.. 14.0001" 2 ok   
```

\subsection{DistanceParametricLineFromParametricLine}

This task ensures the student understands the geometric interpretation of cross product. It is also necessary to understand parametric lines in 3D space.

\subsubsection{Problem formulation}

Find the distance between two 3D lines given parametrically.

\subsubsection{ROPOT question example}

```
Find the distance <m>d</m> between lines <m>{p_1}(t) = 
\begin{pmatrix} 0 \\ -4 \\ -5 \end{pmatrix} + 
t \begin{pmatrix} 0 \\ -4 \\ -4 \end{pmatrix}</m> and 
<m>{p_2}(t) = \begin{pmatrix} -3 \\ 0 \\ 1 \end{pmatrix} + 
t \begin{pmatrix} 0 \\ -2 \\ 2 \end{pmatrix}</m> (if the answer 
is a floating point number round it to 4 decimal places).
<m>d = </m> :n

:n=" 2.9999.. 3.0001" 2 ok   
```

\subsection{DistancePointFromImplicitPlane}

This task ensures the student can apply the formula to compute the point--plane distance.

\subsubsection{Problem formulation}

Find the distance between a point $P$ and a plane given implicitly as $ax + by + cz + d = 0$

\subsubsection{ROPOT question example}

```
Find the distance <m>D</m> from the plane 
<m>3 x - 2 y + 5 = 0</m> to the point <m>[2,3,0]^\top</m>

<m>D = </m> :n

:n=" 1.3867.. 1.3869" 2 ok 
```

\subsection{IntersectionParametricLineImplicitPlane}

To solve this task, the student must understand relative positions of planes and lines in 3D space. It is necessary to understand normal and directional vectors of lines and planes.

\subsubsection{Problem formulation}

Find the intersection of a parametric line and an implicit plane.

\subsubsection{ROPOT question example}

```
Find the intersection point <m>C</m>of the line 
<m>{p}(t) = \begin{pmatrix} 1 \\ 2 \\ 3 \end{pmatrix} + 
t \begin{pmatrix} 4 \\ 1 \\ -3 \end{pmatrix}</m> and 
the plane <m>2 x - 2 y + 2 z + 12 = 0</m>.
If the line is parallel to the plane fill in "||" in 
all the fields. If the line lies in the plane fill in 
"L" in all the fields. (if the answer is a floating 
point number round it to 4 decimal places).

<M>C_1 = </m> :t1
<M>C_2 = </m> :t2
<M>C_3 = </m> :t3

:t1="||" 2 ok
:t2="||" 2 ok
:t3="||" 2 ok
```

\subsection{VectorReflectionOnPlane}

This task ensures the student understands the notion of vector reflections and he can compute it in this particular instance.

\subsubsection{Problem formulation}

Find the reflection of a given vector on a given implicit plane.

\subsubsection{ROPOT question example}

```
Fill in the coefficients of the vector <m>v', \|v'\|=1</m> 
which is the reflection of the vector <m>v = (4,0,-1)^\top</m> 
on the plane <m>5 y - 3 z + 4 = 0</m>. If the vector is 
parallel to the plane fill in "||" in all the fields. 
If the answer is a floating point number 
round it to 4 decimal places.

<M>v'_1 = </m> :n1
<M>v'_2 = </m> :n2
<M>v'_3 = </m> :n3

:n1=" 0.9700.. 0.9702" 2 ok
:n2="-0.2141..-0.2139" 2 ok
:n3="-0.1142..-0.1140" 2 ok
```

\section{Lecture 6}

The sixth lecture provides an introduction to interpolation -- for 1D and 2D signals. It explains basic interpolation methods -- nearest neighbor, linear interpolation, while offering a more in-depth intuition of the presented topic. Interpolation convolution is presented along with various cubic interpolation kernels.

The second part of the lecture focuses on generalizing these techniques for 2D signals.

\subsection{Interpolation1D}

The student should be able to compute Nearest neighbour interpolation, Linear interpolation, Catmull-Rom interpolation, Lanczos with ($n=2$) interpolation. By completing this task, the student can also see a comparison of these interpolation methods.

\subsubsection{Problem formulation}

Compute the following interpolations for a given point in a given grid with known function values: Nearest neighbour interpolation, Linear interpolation, Catmull-Rom interpolation, Lanczos with ($n=2$) interpolation.

\subsubsection{ROPOT question example}

```
The function <m>g(u)</m> is given on the grid 
<m>u=(-5, -4, -3, -2, -1, 0, 1, 2, 3, 4, 5)</m>. 
The corresponding function values are 
<m>g(u_i)=(-43, -28, 4, 20, 13, 0, 0, 7, 2, -17, -29)</m>.
Fill in the value of interpolated function 
<m>\Hat{g}(x_0)</m> where <m>x_0=1.4</m> for different 
interpolation methods. If the answer is a floating point 
number round it to 4 decimal places.

Nearest neighbour interpolation: 
<m>\Hat{g}_{NN}(x_0) = </m> :n1
Linear interpolation: 
<m>\Hat{g}_{lin}(x_0) = </m> :n2
Catmull-Rom interpolation: 
<m>\Hat{g}_{CR}(x_0) = </m> :n3
Lanczos with (n=2) interpolation: 
<m>\Hat{g}_{L2}(x_0) = </m> :n4

:n1="-0.0001.. 0.0001" 1 ok
:n2=" 2.7999.. 2.8001" 1 ok
:n3=" 2.9431.. 2.9433" 3 ok
:n4=" 2.8719.. 2.8721" 3 ok
```

\subsection{InterpolationBilinear}

This tasks tests whether the student is able to extend 1D linear interpolation into a 2D grid.

\subsubsection{Problem formulation}

Compute the bilinear interpolation for a given point inside a given $2\times2$ grid $g$ with known function values.

\subsubsection{ROPOT question example}

```
The function <m>I(u,v)</m> is given on the grid
<m>u=(-3, -2)</m> x <m>v=(4, 5)</m>. The corresponding 
function values are
<m>I(2,2)= 5</m>
<m>I(3,2)= 1</m>
<m>I(2,3)= -4</m>
<m>I(3,3)= -2</m>
Fill in the value of bilinearly interpolated function 
<m>\Hat{I}(x_0,y_0)</m> where <m>x_0=-2.3</m>, <m>y_0=4.2</m>.
If the answer is a floating point number 
round it to 4 decimal places.
<m>\Hat{I}(x_0,y_0) = </m> :n1

:n1=" 1.2399.. 1.2401" 2 ok
```

\subsection{InterpolationCubic}

This tasks makes sure the student is able to use cubic B-spline interpolation for a concrete instance.

\subsubsection{Problem formulation}

Compute the cubic B-spline interpolation for a given point inside a given $4\times4$ grid with known function values.

\subsubsection{ROPOT question example}

```
The function <m>I(u,v)</m> is given on the grid 
<m>u=(2, 5)</m> x <m>v=(-4, -1)</m>. The corresponding 
function values are
<m>I(1,1)= 4</m>
<m>I(2,1)= 2</m>
<m>I(3,1)= 1</m>
<m>I(4,1)= -3</m>
<m>I(1,2)= 3</m>
<m>I(2,2)= 4</m>
<m>I(3,2)= -3</m>
<m>I(4,2)= 5</m>
<m>I(1,3)= -2</m>
<m>I(2,3)= 5</m>
<m>I(3,3)= 3</m>
<m>I(4,3)= 1</m>
<m>I(1,4)= -5</m>
<m>I(2,4)= -5</m>
<m>I(3,4)= 1</m>
<m>I(4,4)= 5</m>
Fill in the value of function <m>\Hat{I}(x_0,y_0)</m> 
approximated by cubic B-spline at the point <m>x_0=3.7</m>, 
<m>y_0=-2.4</m>.If the answer is a floating point number 
round it to 4 decimal places.
    
<m>\Hat{I}(x_0,y_0) = </m> :n1

:n1=" 1.8649.. 1.8651" 2 ok
```

\section{Lecture 7}

This lecture is an introduction to quaternions and rotations. It provides a quick overview of complex numbers and their manipulation and then attempts to generalize complex numbers into quaternions. It explains basic quaternion manipulation -- conjugates, inverses, addition, multiplication, exponentiation.

The lecture also explains the connection of quaternions and rotations, using their trigonometric form.

\subsection{ComplexNumberSimplification}

By solving this task, the student proves his ability to work with complex expressions. This includes division and exponentiation.

\subsubsection{Problem formulation}

Find the real and the imaginary part of a given complex expression.

\subsubsection{ROPOT question example}

```
Compute the complex number <m>z=(a,b)=
\frac{\left(-5 - 2 i\right) \left(-5 + 2 i\right)}
{-3 - 5 i + 5 + 5 i}</m>
If the answer is a floating point number 
round it to 4 decimal places.
<m>a = </m> :n1
<m>b = </m> :n2

:n1=" 14.4999.. 14.5001" 3 ok
:n2="-0.0001.. 0.0001" 3 ok
```

\subsection{QuaternionArithmetic}

This task ensures The student must know quaternion conjugates, inverses, addition, and multiplication. 

\subsubsection{Problem formulation}

Compute simple expression involving two given quaternions.

\subsubsection{ROPOT question example}

```
There are 2 quaternions given: 
<m>q_1=2 + 0 i + \left(-2\right) j + \left(-5\right) k</m> and 
<m>q_2=\left(-5\right) + 2 i + 0 j + \left(-4\right) k</m>
Compute the following expression:
<m>q = q_2^*q_2+3q_1^{-1}q_1^*</m>
If the answer is a floating point number 
round it to 4 decimal places.

<m>q = </m> :n1 <m>+</m> :n2 
<m>i+</m> :n3 <m>j+</m> :n4 <m>k</m>

:n1=" 42.7272.. 42.7274" 3 ok
:n2="-0.0001.. 0.0001" 3 ok
:n3=" 0.7272.. 0.7274" 3 ok
:n4=" 1.8181.. 1.8183" 3 ok
```

\subsection{QuaternionToTrigonometric}

The student is able to convert a quaternion to its trigonometric form.

\subsubsection{Problem formulation}

Compute the trigonometric form of a given rotation quaternion.

\subsubsection{ROPOT question example}

```
There is quaternion given: <m>q=- \frac{\sqrt{2}}{2} + 
\frac{\sqrt{5}}{10} i + 0 j + \frac{3 \sqrt{5}}{10} k, 
q\in \mathbb{H}_1</m>. Fill in the parameters of its trigonometric 
form <m>q = (\cos\theta, {v} \sin\theta), \theta \in (-\pi,\pi]</m>.
If the answer is a floating point number round it to 4 decimal 
places. Enter the solution having the first nonzero component of 
<m>{v}</m> positive.

<m>\theta = </m> :n1
<m>v_1 =</m> :n2
<m>v_2 =</m> :n3
<m>v_3 =</m> :n4

:n1=" 2.3561.. 2.3563" 3 ok
:n2=" 0.3161.. 0.3163" 3 ok
:n3="-0.0001.. 0.0001" 3 ok
:n4=" 0.9486.. 0.9488" 3 ok
```

\subsection{RotateVectorBy2Quaternions}

To solve this task, the student must know how to compose quaternion rotations. 

\subsubsection{Problem formulation}

Rotate a given vector by two quaternions in a given order.

\subsubsection{ROPOT question example}

```
Rotate the vector <m>{v}=(-1, 0, 0)^\top</m> using first 
the quaternion <m>q_1=- \frac{\sqrt{3}}{2} + 
\frac{\sqrt{2}}{4} i + \left(- \frac{1}{4}\right) j + 
\left(- \frac{1}{4}\right) k</m>, then the quaternion 
<m>q_2=0 + \frac{2}{3} i + \frac{1}{3} j + \frac{2}{3} k, 
q_1,q_2\in \mathbb{H}_1</m>. If the answer is 
a floating point number round it to 4 decimal places.

<m>{v'}_1 =</m> :n1
<m>{v'}_2 =</m> :n2
<m>{v'}_3 =</m> :n3

:n1=" 0.5114.. 0.5116" 4 ok
:n2=" 0.1369.. 0.1371" 4 ok
:n3="-0.8484..-0.8482" 4 ok
```

\subsection{RotateVectorByQuaternion}

The student proves that he is able to rotate a vector using quaternions.

\subsubsection{Problem formulation}

Rotate a vector by a given quaternion.

\subsubsection{ROPOT question example}

```
Rotate the vector <m>{v}=(-3, 1, 0)^\top</m> using 
the following quaternion <m>q=\left(- \frac{1}{2}\right) 
- \frac{\sqrt{2}}{2} i + 0 j + \left(- \frac{1}{2}\right) 
k, q\in \mathbb{H}_1</m>. If the answer is 
a floating point number round it to 4 decimal places.

<m>{v'}_1 =</m> :n1
<m>{v'}_2 =</m> :n2
<m>{v'}_3 =</m> :n3

:n1="-2.0001..-1.9999" 4 ok
:n2="-2.0001..-1.9999" 4 ok
:n3="-1.4143..-1.4141" 4 ok
```

\section{Lecture 8}

The eight lecture dives deeper into quaterion arithmetic, it explains the logarithm and general exponentiation. 

The second part of the lecture covers the rotation interpolation using quaternions: LERP, SLERP, SQUAD.

\subsection{QuaternionExponent}

This task ensures the student understands quaternion normalization and integer exponentiation.

\subsubsection{Problem formulation}

Normalize a given quaternion and raise it to a given power.

\subsubsection{ROPOT question example}

```
There is a quaternion given: 
<m>q=- \frac{\sqrt{6}}{2} + \left(- \frac{3}{2}\right) i 
- \frac{3 \sqrt{2}}{4} j + - \frac{3 \sqrt{2}}{4} k</m>
Compute the normalized version <m>q_N</m> of this quaternion 
and the following expression. If the answer is a floating 
point number round it to 4 decimal places.

<m>q_N^6 = </m> :n1 <m>+</m> :n2 
<m>i+</m> :n3 <m>j+</m> :n4 <m>k</m>

:n1=" 0.9999.. 1.0001" 2 ok
:n2="-0.0001.. 0.0001" 2 ok
:n3="-0.0001.. 0.0001" 2 ok
:n4="-0.0001.. 0.0001" 2 ok
```

\subsection{QuaternionExponentDerivative}

To solve this task, the student must understand quaternion exponentiation as a function and be able to find its derivative.

\subsubsection{Problem formulation}

Normalize a given quaternion $q$ and compute the derivative of $q^t$ in a given point $t$.

\subsubsection{ROPOT question example}

```
There is quaternion given: <m>q=\left(-1\right) + 
\frac{1}{2} i + \left(- \frac{1}{2}\right) j - 
\frac{\sqrt{2}}{2} k</m>. Compute the normalized version 
<m>q_N</m> of this quaternion and the following expression.
If the answer is a floating point number 
round it to 4 decimal places.

<m>f'(t) = \frac{d}{dt}\left(q_N^t\right)</m>
<m>f'(5) = </m> :n1 <m>+</m> :n2 
<m>i+</m> :n3 <m>j+</m> :n4 <m>k</m>

:n1=" 1.6660.. 1.6662" 2 ok
:n2=" 0.8329.. 0.8331" 2 ok
:n3="-0.8331..-0.8329" 2 ok
:n4="-1.1782..-1.1780" 2 ok
```

\subsection{QuaternionLerpSlerp}

The purpose of this task is to make sure the students understand LERP and SLERP for rotations and they can use it practice.

\subsubsection{Problem formulation}

Compute LERP and SLERP for two given rotations and an interpolation parameter.

\subsubsection{ROPOT question example}

```
There are 2 rotations given:
1) Rotation by <m>\theta_1=\frac{\pi}{2}</m> around the vector 
<m>{v_1}=(\frac{4}{3}, \frac{4}{3}, \frac{2}{3})^\top</m>
2) Rotation by <m>\theta_2=\pi</m> around the vector 
<m>{v_2}=(\frac{5}{3}, \frac{10}{3}, - \frac{10}{3})^\top</m>
Compute the interpolated rotation parameters 
<m>\theta_h \in [\frac{\pi}{2}, \pi]</m> and 
<m>{v_h} : \|{v_h}\|=1</m> using Lerp and Slerp for 
the given parameter <m>h=0.7</m>.
If the answer is a floating point number 
round it to 4 decimal places.
<b>Pay attention on conditions for <m>\theta_h</m> and 
<m>{v_h}</m> filling in the values!</b>
<p>
Parameters computed using <m>Lerp(q_1, q_2, h)</m> interpolation:
<m>\theta_h = </m> :n1
<m>v_{h1} = </m> :n2
<m>v_{h2} = </m> :n3
<m>v_{h3} = </m> :n4
<p>
Parameters computed using <m>Slerp(q_1, q_2, h)</m> interpolation:
<m>\theta_h = </m> :n11
<m>v_{h1} = </m> :n21
<m>v_{h2} = </m> :n31
<m>v_{h3} = </m> :n41

:n1=" 2.6332.. 2.6334" 2 ok
:n2=" 0.4588.. 0.4590" 2 ok
:n3=" 0.7445.. 0.7447" 2 ok
:n4="-0.4849..-0.4847" 2 ok

:n11=" 2.5884.. 2.5886" 2 ok
:n21=" 0.4695.. 0.4697" 2 ok
:n31=" 0.7498.. 0.7500" 2 ok
:n41="-0.4661..-0.4659" 2 ok
```

\section{Lecture 9}

The topic of this lecture is curve fitting. It derives both linear and nonlinear least squares methods and explains how to compute them for a given set of points. Then, it briefly covers RANSAC and explains its pros. 

\subsection{LeastSquareVerticalTotalPolynomial}

This tasks tests student's understanding of simple interpolations and gives him the opportunity to compare them.

\subsubsection{Problem formulation}

Given a point set of 4 points, fit the points using vertical and total linear least square approximation, then using second-order least square approximation.

\subsubsection{ROPOT question example}

```
The point set is given: <m>[x,y]^\top=[1,-5]^\top,
[4,-2]^\top,[5,-1]^\top,[6,0]^\top</m>.
Fit the line in this point set using vertical and 
total linear least square approximation.
Fit the polynomial of the second order 
using least square approximation.
Fill in the values of parameters of the requested curves.
If the answer is a floating point number 
round it to 4 decimal places. 

Vertical least square approximation: 
<m>f_{LSv} = </m> :n11 <m>x</m> + :n12
Total least square approximation: 
<m>f_{LSt} = </m> :n21 <m>x</m> + :n22
Polynomial least square approximation: 
<m>f_{LSp2} = </m> :n31 <m>x^2</m> + :n32 <m>x</m> + :n33


:n11=" 0.9999.. 1.0001" 3 ok
:n12="-6.0001..-5.9999" 3 ok

:n21=" 0.9999.. 1.0001" 3 ok
:n22="-6.0001..-5.9999" 3 ok

:n31="-0.0001.. 0.0001" 2 ok
:n32=" 0.9999.. 1.0001" 2 ok
:n33="-6.0001..-5.9999" 2 ok
```

\subsection{NonlinearLeastSquares}

The student understands general interpolation and can apply it to a concrete problem.

\subsubsection{Problem formulation}

Given a point set of 3 points and a nonlinear function, fit the given function to the point set using Gauss-Newton method.

\subsubsection{ROPOT question example}

```
The point set is given: <m>[x,y]^\top=[-2.0,-0.7]^\top,
[-1.5,-0.3]^\top,[-1.0,-0.8]^\top</m>. Fit function 
<m>f(x)=p_{1} \sin{\left(p_{2} x \right)} + p_{3} x</m> into 
this point set using least square approximation.
Fill in the values of parameters <p_i> after the first and 
the second iterations of Gauss-Newton method (k=1, k=2) 
if the initial set of parameters is <m>p^0=[1, \pi, 1]^T</m>.
If the answer is a floating point number 
round it to 4 decimal places.

The values after the first iteration (k=1):
<m>p_1 = </m> :n11
<m>p_2 = </m> :n12
<m>p_3 = </m> :n13
The values after the second iteration (k=2):
<m>p_1 = </m> :n21
<m>p_2 = </m> :n22
<m>p_3 = </m> :n23

:n11=" 0.5604.. 0.5606" 3 ok
:n12=" 2.9188.. 2.9190" 3 ok
:n13=" 0.5741.. 0.5743" 3 ok

:n21=" 0.6323.. 0.6325" 3 ok
:n22=" 2.7578.. 2.7580" 3 ok
:n23=" 0.5691.. 0.5693" 3 ok
```

\section{Lecture 10}

This lecture covers the geometry of curves. It is divided into two sections: parametric and implicit curves.

For parametric curves, it explains how to compute their length, normal, tangent, binormal, Frenet frame and curvature. It also provides a geometric intuition for these terms. 

For implicit curves, it explains the nabla operator and its use (and meaning) for implicit curves. It covers the normal and curvature computation of implicitly given curves.

\subsection{ImplicitCurveProperties}

This task ensures the student can compute basic properties of implicit curves -- the unit normal and curvature.

\subsubsection{Problem formulation}

Find the unit normal and curvature of a given implicit curve at a given point.

\subsubsection{ROPOT question example}

```
There is an implicit curve given by the formula: 
<m>x y - x - y^{3} - 2 y^{2} + \frac{1}{2}=0</m>.
Compute the unit normal and curvature of the curve 
at point <m>[x,y]^\top = [-2, 2]^\top</m>:
<m>n(-2, 2)= (</m> :n11, :n12 <m>)</m>
<m>\kappa(-2, 2)= </m> :n2

If the answer is a floating point number 
round it to 4 decimal places.

:n11=" 0.0453.. 0.0455" 2 ok
:n12="-0.9991..-0.9989" 2 ok
:n2=" 0.0025.. 0.0027" 4 ok
```

\subsection{ParametricCurveProperties}

This task ensures the student can compute basic properties of parametric curves in 3D -- their length, curvature and Frenet frame of reference.

\subsubsection{Problem formulation}

Find the length, curvature and Frenet frame of a given parametric curve in 3D at a given point.

\subsubsection{ROPOT question example}

```
There is a parametric curve given: <m>p(t)=
[3 t - 1, - \cos{\left(\frac{t}{2} \right)} - \frac{\pi}{2}, 
\sin{\left(\frac{t}{2} \right)} - 4]^\top</m> 
with <m>t\in [0, \frac{3 \pi}{2}]</m>.
Compute the length of the curve:
<m>L = </m> :n1
Frenet frame and curvature at point <m>t = \pi</m>.
<m>T(\pi) = (</m> :n21, :n22, :n23 <m>)</m>
<m>N(\pi) = (</m> :n31, :n32, :n33 <m>)</m>
<m>B(\pi) = (</m> :n41, :n42, :n43 <m>)</m>
<m>\kappa(\pi) = </m> :n5

If the answer is a floating point number 
round it to 4 decimal places.

:n1=" 14.3321.. 14.3323" 3 ok

:n21=" 0.9863.. 0.9865" 1 ok
:n22=" 0.1643.. 0.1645" 1 ok
:n23="-0.0001.. 0.0001" 1 ok

:n31="-0.0001.. 0.0001" 1 ok
:n32="-0.0001.. 0.0001" 1 ok
:n33="-1.0001..-0.9999" 1 ok

:n41="-0.1645..-0.1643" 1 ok
:n42=" 0.9863.. 0.9865" 1 ok
:n43="-0.0001.. 0.0001" 1 ok

:n5=" 0.0269.. 0.0271" 3 ok
```

\section{Lecture 11}

The last lecture covers both parametric and implicit surfaces. The latter is only briefly mentioned, because the computations are much easier. 

It revises basic concepts and goes through normal, tangent and curvature computation. It gives an introduction to first and second fundamental form and their applications.

\subsection{ParametricSurfaceProperties}

This task ensures the student can compute basic properties of parametric surfaces -- the unit normal, mean curvature, and Gaussian curvature.

\subsubsection{Problem formulation}

Find the normal, mean and Gaussian curvature of a given parametric surface at
    a given point.

\subsubsection{ROPOT question example}

```
There is a parametric surface given: <m>p(t)=
[u \left(- 2 u^{2} + u v + v\right), 3 u^{3} + 3 u v - 3, 
5 u - v^{3} + 2 v^{2} + 4]^\top</m> 
with <m>(u,v)\in [-2,2]\times[-2,2]</m>.
Compute unit normal and both curvatures of 
the surface at point <m>(u,v) = [-2, 2]^\top</m>.
If the answer is a floating point number 
round it to 4 decimal places.

Normal:
<m>n = (</m> :n11, :n12, :n13 <m>)</m>

Gaussian curvature:
<m>K = </m> :n2

Mean curvature:
<m>H = </m> :n3

:n11="-0.6870..-0.6868" 1 ok
:n12="-0.5476..-0.5474" 1 ok
:n13=" 0.4778.. 0.4780" 1 ok

:n2="-0.0001.. 0.0001" 2 ok

:n3="-0.1237..-0.1235" 3 ok
```

\end{markdown*}
\shorthandon{-}


\chapter{Contents}


\end{document}
